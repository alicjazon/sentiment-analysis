\chapter{Wykorzystane narzędzia i technologie}
\label{cha:technologie}

\section{Python 3.5}
\label{sec:python}
Python jest najpopularniejszym językiem programowania w uczeniu maszynowym \textendash \ według raportu firmy Vision Mobile\cite{visionmobile} używa go 57\% programistów i naukowców związanych z przetwarzaniem danych. Na kolejnych miejscach znajduje się C/C++ (używany przez 43\% programistów) oraz Java (41\%). 

Powyższy raport przytacza również konkretne najpopularniejsze obszary zastosowań danych języków. Język Python jest najbardziej powszechny w eksploracji danych z sieci (\textit{ang. web mining}), przetwarzaniu języka naturalnego oraz w analizie wydźwięku tekstu (używany przez 44\% programistów). Dzieje się tak dlatego, że Python to język o przejrzystej składni, kładący nacisk na prostotę i czytelność\cite{python}. Zaletą są również liczne biblioteki, które można wykorzystać do uczenia maszynowego, między innymi PyTorch, TensorFlow, SciKit i Pandas.

\section{TensorFlow}
\label{sec:tensorflow}

Tensorflow to biblioteka powszechnie wykorzystywana w głębokim uczeniu maszynowym. Dużą zaletą jest możliwość stworzenia całej sieci za pomocą zmiennych zastępczych (\textit{ang. placeholder}). Umożliwiają one oddzielenie fazy konfiguracji od fazy ewaluacji, co pozwala na łatwe skalowanie sieci. Biblioteka ta jest niskopoziomowa, ma więc dużo możliwości do dostosowywania sieci do konkretnych potrzeb. 
Posiada również szczegółową dokumentację z przykładami immplementacji\cite{tensorflow}, a z racji dużej popularności, łatwo znaleźć wiele źródeł i tutoriali wykorzystujących TensorFlow.

Przed przystąpieniem do implementacji brana była pod uwagę również biblioteka Keras. Zapewnia ona wysokopoziomowy interfejs programowania aplikacji, co sprawia, że stworzenie sieci LSTM jest proste i nie wymaga wiele kodu. Ograniczona możliwość dostosowania poszczególnych parametrów sieci sprawia jednak, że w opisywanym w tej pracy przypadku użycie biblioteki Keras się nie sprawdza. 

\section{Scikit-learn}
\label{sec:scikit}

Jest to biblioteka przeznaczona do uczenia maszynowego. W pracy został wykorzystany jej moduł \textit{metrics}\cite{scikit}. Posiada on liczne funkcje do ewaluacji modelu oraz mierzenia jakości klasyfikacji. Na stronie biblioteki opisane są również przykłady zastosowania każdej z metryk oraz sposób interpretacji otrzymanych wyników.

\section{Pozostałe narzędzia}
\label{sec:pozostale}
\begin{itemize}
  \item \textbf{NumPy} \textendash \ biblioteka służąca do obliczeń numerycznych.
  \item \textbf{Pyplot} \textendash \ moduł biblioteki Matplotlib służący do sporządzania wykresów.
\end{itemize}


