\chapter{Wstęp}
\label{cha:wstep}

\section{Wprowadzenie}
\label{sec:wprowadzenie}
Istotną częścią naszego codziennego życia jest pozyskiwanie informacji, choć często bardziej interesujące od faktów bywają opinie. Wraz z dynamicznym rozwojem Web 2.0, czyli sieci socjalnej, nastawionej głównie na treści generowane przez użytkowników portali, zwiększa się potrzeba kategoryzacji tychże tekstów. Wiedza o tym, jaka jest publiczna opinia na temat jakiegoś produktu, osoby bądź miejsca może znacznie ulepszyć strategie biznesowe, polityczne i marketingowe. Manualne przetworzenie tak dużej ilości danych jest bardzo czasochłonne, stąd konieczność stosowania systemów do ich automatycznej analizy.

Celem tej pracy jest stworzenie rozwiązania do analizy opinii w krótkich tekstach. W założeniu program ma określać, czy dane zdanie ma wydźwięk pozytywny, negatywny, czy neutralny. Motywacją do zbadania tego właśnie tematu jest chęć praktycznego zastosowania uczenia maszynowego w dziedzinie, która cieszy się popularnością i daje duże możliwości rozwoju.  
\section{Zawartość pracy}
\label{sec:zawartosc}

\begin{itemize}
  \item \textbf{Rozdział 2:} zawiera opis poruszanego problemu wraz z nakreśleniem możliwych podejść do jego rozwiązania. Przedstawiono uzasadnienie wyboru konkretnego podejścia oraz aktualny stan badań w tej dziedzinie (tzw. \textit{state of the art}).
  \item \textbf{Rozdział 3:} przedstawiono w nim źródło, z którego zostały zaczerpnięte dane do uczenia. Opisany jest sposób reprezentacji danych i etykietowania zdań. 
  \item \textbf{Rozdział 4:} opisuje modele teoretycznie możliwe do zastosowania w badanym problemie. Przedstawiony jest krótki opis i analiza tych modeli oraz ostateczna postać wybranego modelu wraz z wzorami i uzasadnieniem, dlaczego najbardziej pasuje do problemu opisywanego w tej pracy.
  \item \textbf{Rozdział 5:} wymieniono w nim użyte technologie i narzędzia wraz z ich zaletami i uzasadnieniem wyboru. 
  \item \textbf{Rozdział 6:} opisuje implementację modelu z rozdziału czwartego. Przedstawiony jest również dobór parametrów wraz z porównaniem wyników dla różnych wartości, a także regularyzacja modelu i korzyści z niej płynące.
  \item \textbf{Rozdział 7:} przedstawione są tu wyniki najlepszego wytrenowanego modelu wraz z wykresami poszczególnych metryk i analizą jakości klasyfikacji.
  \item \textbf{Rozdział 8:} zawiera analizę konkretnych przypadków wraz z wynikiem ich dopasowania. Opisane są zarówno przykłady prawidłowych dopasowań, jak i błędy pierwszego i drugiego rodzaju.
  \item \textbf{Rozdział 9:} przedstawia krótką dyskusję dotyczącą napotkanych problemów. Przedstawiono również możliwości rozwoju programu. 
  \item \textbf{Rozdział 10:} wnioski z pracy w kontekście realizacji jej celu i możliwości dalszego wykorzystania uzyskanych wyników.
\end{itemize}

\section{Aspekt twórczy i badawczy w pracy}
\label{sec:tworczosc}

Rozdziały od szóstego do dziewiątego stanowią część twórczo-badawczą, która opracowuje metody naukowe przedstawione w części teoretycznej pracy. Kod programu będącego przedmiotem analizy w tej pracy został stworzony w oparciu o źródła wymienione w bibliografii.


