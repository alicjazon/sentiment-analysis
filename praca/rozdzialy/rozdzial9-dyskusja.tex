\chapter{Dyskusja}
\label{cha:dyskusja}

Otrzymane wyniki nie są dokładne, ale po przeanalizowaniu konkretnych przypadków można je uznać za zadowalające. Widać, że duży wpływ na klasyfikację miała nadreprezentacja etykiet neutralnych. Możliwe więc, że błędem był podział na trzy klasy, gdyż nie da się przygotować takich danych, by wszystkie trzy etykiety występowały w równym stopniu, bo w zdaniach, szczególnie złożonych, jest wiele słów, które nie wpływają na ich wydźwięk. Zatem w dalszych badaniach warto będzie sprowadzić klasyfikację do problemu binarnego, gdzie uwzględniamy wyłącznie etykiety pozytywne i negatywne.

Problemem napotkanym podczas badań był długi czas obliczeń, co ograniczało w zakresie doboru parametrów, gdyż nie można było wykonać zbyt wielu prób.

W kwestii dalszego rozwoju programu warto przeprowadzić eksperymenty z modyfikacją poszczególnych parametrów, można się również zastanowić nad zmianą modelu sieci. Dużą poprawę może przynieść zwiększenie i zbalansowanie zbioru uczącego.


